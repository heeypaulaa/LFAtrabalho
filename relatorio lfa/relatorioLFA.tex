\documentclass[a4paper,portuguese,12pt]{article}

\usepackage{amsmath}
\usepackage{amsfonts}
\usepackage{amssymb}
\usepackage[brazil]{babel}
\usepackage[utf8]{inputenc}
\usepackage{float}
\usepackage{wrapfig}
\usepackage{graphicx}
\usepackage{subfig}
\usepackage{tablefootnote}
\usepackage{geometry}
\usepackage[table]{xcolor}
\geometry{verbose,a4paper,tmargin=20mm,bmargin=30mm,lmargin=13.5mm,rmargin=13.5mm}

\begin{document}
\begin{center}
{\footnotesize Graduação em Ciência da Computação. Disciplina: Linguagens Formais e Autômatos. Ano 2017. Prof. Walace De Almeida Rodrigues.}\\[0.5cm]
{\Large \textbf {LFA - ALGORITMOS DE CONVERSÃO DE AUTÔMATOS}}\\[0.5cm]
Ana Paula da Silva Cunha - Matrícula :  0011252		$^1$\\,Suena Batista Galoneti -  Matrícula :	0011251		$^2$\\
$^1$Instituto Federal de Minas Gerais, Formiga, MG\\[1cm]
\end{center}

\begin{quote}
{\small \textbf{Resumo:} Este documento apresenta o relatório do desenvolvimento do primeiro trabalho da disciplina que objetivava a implementação de alguns algoritmos.} \\[0.5cm]
{\small Palavras-Chaves: LFA; Algoritmos.}
\end{quote}
	
\section{INTRODUÇÃO}

\paragraph{} O conceito de autômato o define como um conjunto finito de estados. 

\paragraph{} Autômatos também podem ser definidos através de uma quintupla : Conjunto de Estados, Alfabeto, Função de Transição, Estado Inicial(se determinístico, e Conjunto de Estados Iniciais se não determinístico) e Conjunto de Estados Finais.

\paragraph{}Descrevendo seu funcionamento ou caminhamento vemos que um autômato parte de um estado inicial e caminha para um próximo estado de acordo com a função de transição e a palavra que está sendo reconhecida, e está deve conter apenas elementos do alfabeto do autômato para ser aceita.

\subsection{EQUIVALÊNCIA}

\paragraph{} A equivalência de autômatos se dá a partir da equivalência de seus estados, sendo o principal o estado inicial, pois partindo de uma mesma entrada o comportamento dos dois autômatos deve ser o mesmo para que estes sejam equivalentes.

\subsection{MINIMIZAÇÃO}

\paragraph{} Um autômato é possível de minimização se este possuir estados equivalentes, ou seja, se dois estados pertencentes ao autômato desempenharem o mesmo papel, se este caso ocorrer o autômato mínimo irá apresentar apenas um desses estados.

\subsection{MULTIPLICAÇÃO}

\paragraph{} Utilizamos a multiplicação de autômatos quando, de maneira bruta, quisermos transformar dois autômatos em um. Ou seja, partindo de dois autômato distintos e pré definidos, realizarmos uma operação que ao final resulte em um único autômato. Essas operações podem ser :

\begin{itemize}
\item Complemento
\item Diferença
\item Interseção
\item União
\end{itemize}

\section{DESENVOLVIMENTO}

\paragraph{} Foi optado pelo desenvolvimento dos algoritmos acima descritos na linguagem Java.

\paragraph{} Para a montagem dos autômatos e saída gráfica está sendo utilizado o JFlap.

\paragraph{} Inicialmente foram criadas duas classes nomeadas como, a primeira, de AFD e, a segunda, FunçãoTransição.

\paragraph{} Dentro da classe AFD foi implementado o método para leitura (utilizando árvore) do arquivo XML.

\paragraph{}Outro método presente nesta classe é o para a obtenção do complemento, aonde verifica quais estados são finais e quais não são e transforma os estados não finais nos novos estados finais.

\paragraph{}Para a implementação  da união, interseção e diferença de autômatos fez necessário a implementação da multiplicação. A multiplicação consistiu em unir os alfabetos dos dois autômatos em um novo alfabeto e fazer um novo caminhamento verificando os nós visitados e os a visitar e utilizando da função transição obtendo novos estados para o autômato final(o autômato final terá seus estados finais definidos de acordo com a operação escolhida) e estes estados também devem passar pela verificação de visitados ou a visitar e serem caminhados.

Os estados serão marcados com finais de acordo com a operação : se for união os estados marcados como finais serão as combinações dos estados finais originais, em outras palavras, os estados novos começados com o estado final do primeiro autômato e os novos estados terminados com o estado final do segundo autômato. No caso da interseção serão finais apenas os novos estados que forem compostos exclusivamente por estados finais dos autômatos originais. A diferença nada mais é do que E NÃO, sendo assim ela irá aceitar como estados finais novos apenas os estados começados pelo estados finais do primeiro autômato e terminado pelos estados não finais do segundo autômato.


\paragraph{} Na implementação ainda encontram-se outros pequenos métodos com sub tarefas utilizadas para a implementação das operações do trabalho.



\begin{thebibliography}{1}	
	
	\bibitem[1]{}Ferreira, B.\\
	Introdução a linguagem Java\\
	Programação Orientada a Objetos\\
	Prof. Dr. Bruno Ferreira		
	
\end{thebibliography}


\end{document}