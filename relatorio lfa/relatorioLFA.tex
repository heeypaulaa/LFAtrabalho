\documentclass[a4paper,portuguese,12pt]{article}

\usepackage{amsmath}
\usepackage{amsfonts}
\usepackage{amssymb}
\usepackage[brazil]{babel}
\usepackage[utf8]{inputenc}
\usepackage{float}
\usepackage{wrapfig}
\usepackage{graphicx}
\usepackage{subfig}
\usepackage{tablefootnote}
\usepackage{geometry}
\usepackage[table]{xcolor}
\geometry{verbose,a4paper,tmargin=20mm,bmargin=30mm,lmargin=13.5mm,rmargin=13.5mm}

\begin{document}
\begin{center}
{\footnotesize Graduação em Ciência da Computação. Disciplina: Linguagens Formais e Autômatos. Ano 2017. Prof. Walace De Almeida Rodrigues.}\\[0.5cm]
{\Large \textbf {Algoritmos de Conversão de Autômatos}}\\[0.5cm]
Ana Paula da Silva Cunha - Matrícula: 0011252$^1$\\Suena Batista Galoneti - Matrícula: 0011251$^1$\\
$^1$Instituto Federal de Minas Gerais, Formiga, MG\\[1cm]
\end{center}

\begin{quote}
{\small \textbf{Resumo:} Este documento apresenta o relatório do desenvolvimento do primeiro trabalho da disciplina de Linguagens Formais e Autômatos (LFA) que tem como objetivo implementar algoritmos para manipulação de autômatos finitos determinísticos, a partir de um documento XML .} \\[0.5cm]
{\small Palavras-Chaves: LFA; Algoritmos.}
\end{quote}
	
\section{Introdução}
\paragraph{}O conceito de autômato finito, o define como um conjunto finito de estados. Um AFD (Autômato Finito Determinístico) é uma Quíntupla (\textit{E}, $\Sigma$, $\delta$, \textit{i}, \textit{F}), tal que:
\begin{itemize}
	\item \textit{E}: um conjunto finito não vazio de estados;
	\item $\Sigma$: um alfabeto;
	\item $\delta$: E $\times$ $\sigma$ $\rightarrow$ E é a função de transição, uma função total;
	\item \textit{i}: é o estado inicial que pertence à E;
	\item \textit{F}: é o conjunto de estados finais.
\end{itemize} 

Descrevendo seu funcionamento ou caminhamento vemos que um autômato parte de um estado inicial e caminha para um próximo estado de acordo com a função de transição e a palavra que está sendo reconhecida, e esta deve conter apenas elementos do alfabeto do autômato para ser aceita.

\section{Desenvolvimento}
\paragraph{}Foi sugerido que os algoritmos para manipulação de AFDs, fossem feitos na linguagem de programação Java ou Python, deste modo foi optado pela linguagem Java.

Para a montagem dos autômatos e saída gráfica foi utilizado o JFlap. Este programa tem como extensão um arquivo .jff, que ao abri-lo em formato texto, tem formato XML, sendo assim uma maneira fácil de manipula-lo.  

Inicialmente foram criadas três classes nomeadas como:
\begin{itemize}
	\item TuplaEstados;
	\item FuncaoTransicao;
	\item AFD.
\end{itemize}

Estas funções serão descritas nas subseções a seguir.

\subsection{Classe: \textit{TuplaEstados}}
\paragraph{}Classe feita para manipular estados 'visitados' e à 'visitar' em uma multiplicação de dois autômatos determinísticos. Ao criar um objeto do tipo \textit{TuplaEstados}, seu conteúdo seria estado do primeiro automato e do segundo (afd1E, afd2E). Seus métodos são básicos para manipulação de um objeto, \textit{Geters}, \textit{Seters} e \textit{Override} do método \textit{toString()}.

\subsection{Classe: \textit{FuncaoTransicao}}
\paragraph{}A classe \textit{FuncaoTransicao}, foi feita para manipular cada função de transição do autômato. O objeto do tipo \textit{FuncaoTransicao} é constituído de: o estado de partida do autômato, o estado de chegada do mesmo autômato e uma letra do alfabeto deste (de, para, caracter).

\subsection{Classe: \textit{AFD}}
\paragraph{}Dentro da classe \textit{AFD} foi implementado o método para leitura do arquivo XML, montando um objeto do tipo \textit{AFD}, tendo os seguintes parâmetros para sua construção: \textit{ArrayList} do tipo \textit{Integer} para o conjunto de estados e conjunto de estados finais do autômato, \textit{ArrayList} do tipo \textit{Character} para o alfabeto do AFD, \textit{ArrayList} do tipo \textit{FuncaoTransicao} para todas as funções de transições do autômato, e o estado inicial do tipo \textit{int}.

Outro método presente nesta classe é para a obtenção do complemento, aonde verifica quais estados são finais e quais não são, e transforma os estados não finais nos novos estados finais e os finais anteriores tornando-os não finais.

Para a implementação  da união, interseção e diferença de autômatos, fez necessário a implementação do método \textit{multiplicacao}. Este método consistiu em unir os alfabetos dos dois autômatos em um novo alfabeto e fazer um novo caminhamento verificando os nós "visitados" e os nós à "visitar" (um \textit{ArrayList} do tipo \textit{TuplaEstados} foi feito para cada um), utilizando a função transição obtendo novos estados para o autômato final (o autômato final terá seus estados finais definidos de acordo com a operação escolhida).

A partir da lista de "visitados" obtida na multiplicação, os novos estados serão renomeados e os finais serão marcados de acordo com a operação: 
\begin{itemize}
	\item União: os estados marcados como finais serão as combinações dos estados finais originais, em outras palavras, os estados novos, obtidos numa lista de "visitados", começados com o estado final do primeiro autômato e terminados com o estado final do segundo autômato;
	\item Interseção: serão finais apenas os novos estados que forem compostos exclusivamente por estados finais dos autômatos originais;
	\item Diferença: nada mais é do que a interseção do primeiro automato com o complemento do segundo, ou seja, ela irá aceitar como estados finais novos apenas os estados começados pelo estados finais do primeiro autômato e terminado pelos estados não finais do segundo autômato.
\end{itemize}

Foi implementado um método para cada item anteriormente descrito.

Na implementação ainda encontram-se outros pequenos métodos com sub tarefas utilizadas para a implementação das operações do trabalho, além das requeridas na especificação do trabalho, como:

\begin{itemize}
\item existeNaLista: verifica se existe um objeto em uma lista do tipo \textit{TuplaEstados}
\item getIndex: pega o índice do objeto de uma lista do tipo \textit{TuplaEstados} 
\end{itemize}

\section{Conclusão}
Algumas das funções que foram requeridas na especificação, não foram implementadas, tais como, equivalencia de estados, equivalencia de automatos, minimização.
\bibliographystyle{abnt-alf}
\bibliography{referencia}

\end{document}